%!TEX root = ../agi_mfws414ali.tex
\section{Elementares Konzept und Randbedingungen}
\label{concept}

\subsection{Zielsetzung}
Das Ziel des AntMe!-Projektes war es, ein Ameisenvolk zu entwickeln, welches nicht nur im Einzelspieler, sondern auch im Mehrspieler-Modus möglichst hohe Punktzahlen erreicht.

\subsection{Vorgehensweise}
Diese Ausarbeitung ist in vier Kapitel untergliedert. Im zweiten Kapitel liegt der Fokus auf der Schaffung von Grundlagen für ein allgemeines Verständnis der Thematik. Bei dem dritten Kapitel handelt es sich um den Hauptteil, der Dokumentation des Ameisenvolkes. Abschließend wird in Kapitel vier das erarbeitete Ergebnis zusammenfassend betrachtet und ein Ausblick zu möglichen Verbesserungen gegeben.

\subsection{AntMe!}
AntMe! lässt sich in die Kategorie der Programmierspiele einordnen. Hierbei handelt es sich um eine spezielle Ausprägung von Computerspielen, welche während der Spielpartie keine Interaktion von dem User zulassen. Der Spieler muss hingegen das Verhalten der Spielfigur programmieren.\footcite[Vgl.][]{Wikipedia}

Bei AntMe! sind die Spielfiguren Ameisen, jedoch hat der Spieler nicht nur Einfluss auf das Verhalten einer einzigen Ameise sondern ist für die Steuerung eines gesamten Ameisenvolkes zuständig. Dem Spieler ist es möglich das Verhalten einer Ameise des Ameisenvolkes innerhalb eines vorgegebenen Rahmens durch Programmieren exemplarisch zu bestimmen. Jede Ameise des Ameisenvolkes verhält sich exakt nach dem gleichen Programmablauf, daher ist es wichtig, dass die Ameisen möglichst flexibel programmiert werden. Der Programmablauf von AntMe! ist ereignisgesteuert, sodass sich die Ameise immer entsprechend der aktuellen Situation verhalten kann. Weiterhin gibt es einen Mechnismus, der eine Individualisierung der Handlungsweise einer Ameise  erlaubt. Hierbei handelt es sich um sogenannte Kasten, denen eine Ameise zugehören kann. Eine Kaste zeichnet sich durch eine Spezialisierung in bestimmten Bereichen aus. Zudem ist es möglich, für jede Kaste eine spezifische Verhaltensweise zu programmieren.

Das Spielkonzept von AntMe! ist der Realität nachempfunden. Als Ausgangssituation bei AntMe! startet jedes Ameisenvolk an seinem Ameisenbau. Ein Ameisenvolk kann (standardmäßig) maximal aus 100 Ameisen bestehen. Die Ameisen werden in einem festgelegten Abstand geboren. Weitere Spielelemente sind bspw. Zucker und Äpfel, die von den Ameisen gesammelt und zurück in den Bau gebracht werden können, um Punkte zu erhalten. Die erhaltenen Punkte sind beim Zucker abhängig von der Maximallast einer Ameise, diese ist durch Spezialisierung\footnote{Die Ameisen können bspw. auf das Sammeln von Nahrung spezialisiert werden. Diese Methodik wird in Zusammenhang mit der Strategie in Kapitel 3 näher betrachtet.} änderbar. Wie in der Natur haben auch die Ameisen in AntMe! einen natürlichen Gegenspieler, dieser wird in Form von Wanzen dargestellt. Wanzen sind dauerhaft auf der Suche nach Ameisen, um diese zu töten. Ameisen können diese ebenfalls, wie auch gegnerische Ameisen, angreifen und töten, wodurch Punkte generiert werden können. Stirbt eine Ameise durch eine andere, feindliche Ameise, so bekommt der Angreifer zusätzliche Punkte und dem Verlierer werden Punkte abezogen.

\begin{table}[hbt]
\centering
\begin{minipage}[t]{.6\textwidth} % Breite der Tabelle		
\caption{Punktevergabe in AntMe! im Überblick} % Überschrift
\begin{tabularx}{\columnwidth}{rX}
\toprule
Art & Punkte\\
\midrule
Apfel & +250\\
Zucker & +4 bis +10\\
Wanze & +150\\
Feindl. Ameise & +5\\
Tod durch feindl. Ameise & -5\\
\bottomrule
\end{tabularx}
\source{Eigene Darstellung in Anlehnung an \cite{AntMeWiki3}} % Quelle
\label{tab:points}
\end{minipage}
\end{table}

Des Weiteren ist zwischen dem Programm AntMe! und dem AntMe!-Framework zu unterscheiden. Bei der Anwendung von AntMe! handelt es sich um das eigentliche Spiel, also die Umgebung in der die programmierten Ameisenvölker in Einzel- oder Mehrspieler-Partien antreten können. Das AntMe!-Framework hingegen stellt die grundlegenden Funtionalitäten für die Programmierung der Ameisen dar. AntMe! ist in C\# programmiert und kann daher in den von dem .Net-Framework unterstützten Sprachen programmiert werden.