%!TEX root = ../agi_mfwis415af4l.tex
\section{Fazits aller Teammitglieder}
\label{concept}

\subsection{Jannik Maes}\\
Zusammenfassend lässt sich sagen, dass in die Entwicklung einer App neben Zeit auch viel Disziplin und Ehrgeiz gesteckt werden muss, damit das Projekt am Ende erfolgreich ist. Zudem sollte man eine klare Vorstellung vom Endergebnis im Kopf haben und sich persönliche Ziele setzen, welche man im Verlauf des Projektes erreichen möchte. Von kleinen Problemen oder Tiefschlägen bei der Programmierung darf man sich nicht unterkriegen lassen sondern muss stets das Ziel und den Projekterfolg für das gesamte Team vor Augen halten. So können kleine Erfolge bei der Programmierung anstacheln und motivieren, beispielsweise wenn die Activity nach einigen frustrierenden Stunden endlich fehlerfrei funktioniert und den persönlichen sowie gewünschten Vorstellungen entspricht. Mithilfe solcher Erfolge habe ich für mich persönlich neuen Aufschwung und Freude am Projekt gewonnen, sodass ich meine Zeit gerne in das Projekt sowie den Projekterfolg investiert habe. Die Zusammenarbeit mit meinen Teammitgliedern hat mir sehr viel Freude bereitet, was auch prägend für unseren Projekterfolg als Team war und ist. Der Aspekt Kommunikation im Team war sehr wichtig für uns, um etwaige Unstimmigkeiten oder Problemen während der Programmierung der App zu vermeiden. Daher habe ich gerne zahlreiche Teammeetings einberufen und unser motiviertes Team über die Wochen mit viel Freude koordiniert. Wir haben als Team sehr gut zusammengearbeitet und uns gegenseitig unterstützt, wenn es zu Problemen bei der Programmierung an der ein oder anderen Stelle kam. Auf unsere erste eigene App die 
\textit{food4life} App sind wir als Team sehr stolz und ich persönlich bin auch sehr stolz und zufrieden mit meiner eigenen Programmierleistung. Durch das Projekt konnte ich für mich persönlich viele schöne und lehrreiche Erfahrungen mitnehmen, welche ich fortan in zukünftigen Projekten einsetzen werde.

\\
\subsection{Sven Kuczera}\\
Das Projekt 
\textit{food4life} hat viel Zeit und Mühe gekostet, das gesamte Team hat viele lange Stunden an der App und dem allgemeinen Projekt gearbeitet, trotzdem war es für alle ein erfreuliches Arbeiten und vor allem Ergebnis der App. Persönlich lässt sich sagen das ich die Projektzeit auch wenn sie äußerst anstrengend war gut in Erinnerung behalten werde, was auch an der extrem guten Teamdynamik liegt. Wir haben uns gegenseitig unterstützt und geholfen so oft es ging. Der Fortschritt in dem Projekt war stetig, was einer guten Planung zu danken war. Die Arbeitsverteilung habe ich als sehr gerecht und fair empfunden. Die Kommunikation innerhalb der Gruppe hat sich sehr positiv auf das allgemeine Projekt ausgeübt, sodass keine Umstände und Unstimmigkeiten aufgetreten sind. Bei dem allgemeinen Design und dem Aufbau der App waren wir einstimmig, was das Projekt beschleunigt hat. Die vielen Teammeetings haben sich als ausschlaggebend für ein besseres Verständnis der App und des Projekts herauskristallisiert. Das knobeln an diversen Hürden und Hindernissen hat sich als hilfreich für den gesamten Entwicklungsverlauf der App herausgestellt. Die Hürden die vor allem der fehlenden Programmiererfahrung mit Java zu Grunde lagen habe ich meistern können, was jedoch sehr Zeitaufwändig war. Gegen Ende der App waren die Probleme allerdings beiseitegelegt, sodass hier keine Schwierigkeiten mehr auftraten. Auch wenn es zwischendurch nicht immer so aussah und auch wenn man zwischendurch kurz vor dem aufgeben war, haben wir am Ende eine schöne und voll funktionale App entwickeln können, was noch einmal klargestellt hat, dass sich die ganze Mühe gelohnt hat. Abschließend kann ich sagen das ich das Team, die Entwicklung der App und das ganze Projekt als extrem positiv einordne.

\\
\subsection{Henryk Schaffrath}\\
Das Projekt
\textit{food4life}  hat viel Zeit und Aufwand gekostet. Dennoch hat sich der Aufwand und das Engagement des Teams gelohnt und haben so eine wirklich schöne und gute Applikation für das „Kalorien zählen“ während des Tages erstellt. Die App hat einen ständigen Zuwachs über die Wochen erfahren. Dies verdanken wir der guten Planung und der professionellen Umsetzung dessen. Die Arbeit in unserem Team hat mir sehr gut gefallen. Die häufigen Teammeetings haben dazu beigetragen, dass wir wenig Unstimmigkeiten innerhalb des Teams hatten. Dennoch habe ich festgestellt, dass ich ein paar Defizite beim Programmieren hatte und mir einige Fähigkeiten entweder durch Internetvideos oder durch meine Teammitglieder angeeignet habe, dies war jedoch sehr zeitaufwändig. 
Allgemein ist zu sagen, dass mir die Arbeit in meinem Team sehr viel Spaß bereitet hat und ich das Arbeitsklima sehr positiv empfunden habe, weswegen das Projekt zu so einem erfolgreichen Endergebnis geführt hat. Mir wird die Zeit positiv im Kopf bleiben, da wir als Team sehr viel geleistet haben und das Projekt erfolgreich abgeschlossen haben.

\\
\subsection{Benedikt Burczek}\\

\\