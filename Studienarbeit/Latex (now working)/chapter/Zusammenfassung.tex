%!TEX root = ../agi_mfws414ali.tex
\section{Zusammenfassung}

Das zu Anfang gesetzte Ziel, eine Strategie zu entwerfen und umzusetzen, mit der hohe Punktzahlen sowohl im Einzelspieler als auch im Mehrspieler-Modus möglich sind, konnte erfolgreich erreicht werden. Für die Überprüfung, dass dieses Ziel erreicht werden konnte, wurden diverse repräsentative Messreihen durchgeführt.

Bei der Entwicklung einer zielführenden Strategie sind verschiedene Probleme aufgetreten, die gelöst werden mussten, wie bspw. die Notwendigkeit, dass über eine Duftmarke verschiedenartige Informationen geteilt werden können. Zudem sind weitere Probleme aufgetreten, die einem nicht-statischen Ameisenvolk vorbehalten sind, da diese im Vergleich zu statischen relativ stark eingeschränkt sind in den Umsetzungsmöglichkeiten. Das geht so weit, dass bpsw. das Ereignis, dass eine Ameise gestorben ist, nur von statischen Ameisen zu gebrauchen ist. Die größten Probleme traten bei dem effektiven Einsatz von verschiedenen Kasten auf. Im Rahmen dieses Projektes hat sich herausgestellt, dass statische Ameisenvölker das größte Potenzial hinsichtlich der Kasten-Mechanik aufweisen. Statische Völker sind dazu fähig die eingesetzten Kasten der vorliegenden Spielsituation anzupassen. Weiterhin erwies sich die Realisierung von Gruppierungen als problematisch. Mit Gruppierungen sind nicht die unterschiedlichen Kasten gemeint (siehe aTomGruppenAmeisen), sondern das Gruppieren von Ameisen zu Fünfer- oder Zehner-Gruppen. Diese Strategie wurde zeitweise verfolgt, jedoch wieder verworfen, da sich dies nicht erfolgreich nutzen lies. Auch hier stößt ein nicht-statisches Ameisenvolk an die Grenzen. Als Fazit zu einem nicht-statischen Ameisenvolk lässt sich sagen, dass die größte Schwierigkeit, die es zu meistern gilt, die Beschränkungen sind. Generell lassen sich mit statischen Ameisen komplexere Strategien einfacher umsetzen.

Des Weiteren gab es auch einige Herausforderungen bei der Programmierung eines Ameisenvolkes. Das Entwickeln einer eigenen Basis-Klasse, welche die Basis-Klasse von AntMe! erweitert ist nicht möglich. Auch die Verwendung von Konstanten und Enumerationen war nicht möglich, da das Ameisenvolk sonst als statisch gekennzeichnet wurde. Allerdings spiegelt dies auch den Praxisalltag wieder, wo man des Öfteren ebenfalls mit Einschränkungen umgehen muss, wenn man bspw. Frameworks oder API's einsetzt.