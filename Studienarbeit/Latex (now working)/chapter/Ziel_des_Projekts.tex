%!TEX root = ../agi_mfws414ali.tex
\section{Ziel des Projekts}
\label{concept}
Das Ziel des Projektes Food4Life ist es ein neuartiges Kalorientagebuch als Android Applikation zu entwickeln. Die App hilft dem Benutzer die konsumierten Lebensmittel und deren enthaltende Kalorien in einem Tagebuch zu erfassen. Außerdem wird dem Benutzer ein auf ihn persönlich zugeschnittenes Tagespensum angezeigt. Mit der einfacheren und übersichtlichen Oberfläche und dem äußerst angenehmen Layout und dem intuitiven Interaktionsverhalten bietet die App von Food4Life eine überraschend andere Erfahrung bei der Verwendung eines „Calorie Trackers“. Zur Vereinfachung der Benutzung ist es außerdem möglich verschiedene Lebensmittel zu einem Menü hinzuzufügen und diese dann mit einem Klick in das Tagebuch einzutragen. Die App bietet zudem die Möglichkeit der statistischen Auswertung des durchschnittlichen Kalorienverbrauchs der letzten 7, 14 und 30 Tage. 

